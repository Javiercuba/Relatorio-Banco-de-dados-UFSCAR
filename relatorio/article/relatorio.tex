% ------------------------------------------------------------%
% 2020-12-05
% 2015-2020 - Emerson Ribeiro de Mello - mello@ifsc.edu.br
% ------------------------------------------------------------%
\documentclass[11pt]{../../classes/ifscarticle}
\usepackage{../../classes/ifscutils}
\usepackage[alf]{abntex2cite} % Citações padrão ABNT

\renewcommand*{\theusecase}{UC-\thesection.\arabic{usecase}}


\usepackage{lipsum}

\AtBeginDocument{\thispagestyle{empty}}
\begin{document}

% ------------------------------------------------------------%
% Capa
% -------1-----------------------------------------------------%
\begin{center}

    {\large Universidade Estadual Norte Fluminense}\\[0.2cm] %0,2cm é a distância entre o texto dessa linha e o texto da próxima
    {\large Banco de Dados - Sahudy }\\[0.2cm] % o comando \\ "manda" o texto ir para próxima linha
    {\large Ciência da Computação}\\[5.2cm]
    

% Título
{\Huge \bfseries Tema 18 - Pescarte}

\vspace{.5cm}

% Subtítulo
{\LARGE \bfseries Fase Intermediária}

\vfill
\end{center}
\begin{tabbing}
   
\end{tabbing}

{\noindent \large \bfseries 
Javier Ernesto 
\\[.5em] Matheus de Sousa  
}


\begin{flushright}
Data de entrega: 29 de mar\c{c}o de 2021
\end{flushright}

\clearpage
\pagestyle{firstpage}
% ------------------------------------------------------------%

% ------------------------------------------------------------%
% Adicionando sumário
% ------------------------------------------------------------%
\tableofcontents
\clearpage

% ------------------------------------------------------------%
% Início do documento
% ------------------------------------------------------------%

\section{Introdução}
\label{sec:introducao}

Na introdução é apresentada a organização do projeto e uma breve descrição de como será construído. Esse é um modelo em e o estilo das referências bibliográficas segue as normas da ABNT, implementadas pelo pacote abnTeX2. A \autoref{fig:logolatex} apresentada o logo do \LaTeX.

\begin{figure}[ht]
    \centering
    \includegraphics[width=.5\linewidth]{figuras/logoPescarte}
    \caption{Logo do Pescarte}
    \label{fig:logolatex}
\end{figure}

O restante do documento está organizado da seguinte forma. Na \autoref{sec:escopo} é apresentado o escopo do projeto.

\subsection{Escopo do projeto}
\label{sec:escopo}

Indica o propósito do sistema a ser desenvolvido. Por exemplo, descreve a necessidade que foi apresentada pela área requisitante. \lipsum[2]


\section{Descrição do problema}


Nessa seção deve ser apresentada a lista de pessoas que participaram do levantamento de requisitos. Pode-se basear nos livro de \cite{bezerra02}.

\lipsum[2]


\subsection{Consultas}

\begin{enumerate}
    \item O sistema deve permitir que alunos visualizem as notas obtidas
    \item O sistema deve permitir que os alunos façam matrícula nas disciplinas do semestre letivo
    \item O sistema deve permitir que os alunos possam obter seus históricos escolares
\end{enumerate}

\subsubsection{“quais são os funcionários que trabalham em cada departamento?”}

Nessa seção deve ser apresentada a lista de pessoas que participaram do levantamento de requisitos.


\subsubsection{“quais são os funcionários que trabalham em cada departamento?”}

Nessa seção deve ser apresentada a lista de pessoas que participaram do levantamento de requisitos.


\subsubsection{“listar todos os funcionários que trabalham no departamento de TI”}


\subsubsection{“quais são os funcionários que trabalham em cada departamento?”}


\subsubsection{“quais são os funcionários que trabalham em cada departamento?”}


\subsubsection{“quais são os funcionários que trabalham em cada departamento?”}


\subsubsection{“quais são os funcionários que trabalham em cada departamento?”}


\subsubsection{“quais são os funcionários que trabalham em cada departamento?”}


\subsection{Regras de negócio}
\label{sec:regrasdenegocio}

\begin{enumerate}
    \item \textbf{Número máximo de matrículas por semestre letivo}
    \begin{itemize}
        \item Em um semestre letivo um aluno não poderá se matricular em um número de disciplinas cujo o soma de créditos ultrapasse 18.
    \end{itemize}
    \item \textbf{Número máximo de alunos por turma}
    \begin{itemize}
        \item Uma oferta de disciplina não pode ter mais de 40 vagas para matrícula.
    \end{itemize}
    \item \textbf{Validação de créditos}
    \begin{itemize}
        \item Um aluno só poderá solicitar validação de uma disciplina caso nunca tenha se matriculado na mesma.
    \end{itemize}
\end{enumerate}


\section{Projeto Conceitual}


\begin{usecase}{Consultar informações}{uc:consultarinfos}{Esse caso de uso descreve as etapas para o PAF obter informações como versão do SB, resumo criptográfico do SB, IdDAF, modelo, fabricante, valor atual do contador, certificado armazenado, estado do DAF, algoritmos criptográficos que o DAF provê suporte e identificadores dos documentos retidos na MT.}
    \begin{cabecalhoUC}
    \atorPrimario{PAF}
    % \atoresSecundarios{ABC, DEF}
    \precondicoesUC{DAF deve estar no estado \textsc{Pronto}, \textsc{Inativo} ou \textsc{Bloqueado}}
    \end{cabecalhoUC}

    \begin{fluxoprincipal}
        \item O PAF solicita ao DAF suas informações 
        \item O DAF retorna para o PAF o documento estruturado com suas informações
    \end{fluxoprincipal}  

    \begin{fluxoexcecao}{Pedido mal formado}
        \item O DAF retorna para o PAF uma mensagem de erro informando que o pedido foi mal formado)
    \end{fluxoexcecao}   
\end{usecase}


% ----------------------------------------------------------
% Referências bibliográficas
% ----------------------------------------------------------
\bibliography{referencias}


\end{document}

