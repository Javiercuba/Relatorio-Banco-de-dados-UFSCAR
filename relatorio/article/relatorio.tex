% ------------------------------------------------------------%
% 2020-12-05
% 2015-2020 - Emerson Ribeiro de Mello - mello@ifsc.edu.br
% ------------------------------------------------------------%
\documentclass[11pt]{../../classes/ifscarticle}
\usepackage{../../classes/ifscutils}
\usepackage[alf]{abntex2cite} % Citações padrão ABNT

\renewcommand*{\theusecase}{UC-\thesection.\arabic{usecase}}


\usepackage{lipsum}

\AtBeginDocument{\thispagestyle{empty}}
\begin{document}

% ------------------------------------------------------------%
% Capa
% -------1-----------------------------------------------------%
\begin{center}

    {\large Universidade Estadual Norte Fluminense}\\[0.2cm] %0,2cm é a distância entre o texto dessa linha e o texto da próxima
    {\large Banco de Dados - Sahudy }\\[0.2cm] % o comando \\ "manda" o texto ir para próxima linha
    {\large Ciência da Computação}\\[5.2cm]


    % Título
    {\Huge \bfseries Tema 18 - PEA Pescarte}

    \vspace{.5cm}

    % Subtítulo
    {\LARGE \bfseries Fase Intermediária}

    \vfill
\end{center}
\begin{tabbing}

\end{tabbing}

{\noindent \large \bfseries
Javier Ernesto
\\[.5em] Matheus de Sousa
}


\begin{flushright}
    Data de entrega: 29 de mar\c{c}o de 2021
\end{flushright}

\clearpage
\pagestyle{firstpage}
% ------------------------------------------------------------%

% ------------------------------------------------------------%
% Adicionando sumário
% ------------------------------------------------------------%
\tableofcontents
\clearpage

% ------------------------------------------------------------%
% Início do documento
% ------------------------------------------------------------%

\section{Introdução}
\label{sec:introducao}

O Projeto PESCARTE tem como sua principal finalidade a criação de uma rede social regional integrada por pescadores artesanais e por seus familiares, buscando, por meio de processos educativos, promover, fortalecer e aperfeiçoar a sua organização comunitária e a sua qualificação profissional, bem como o seu envolvimento na construção participativa e na implementação de projetos de geração de trabalho e renda.
\begin{figure}[ht]
    \centering
    \includegraphics[width=.5\linewidth]{figuras/logoPescarte}
    \caption{Logo do Pescarte}
    \label{fig:logolatex}
\end{figure}

\section{Descrição do problema}

O PEA PESCARTE é divido por quatro núcleos:
\begin{itemize}
    \item Núcleo A
    \item Núcleo B
    \item Núcleo C
    \item Núcleo D
\end{itemize}
Esses Núcleos são compostos por 21(Vinte e uma) linhas
de pesquisa que por sua vez são formadas por
diversos pesquisadores. Um pesquisador possui três tipos:
\begin{itemize}
    \item Pesquisador de Iniciação científica
    \item Pesquisador de mestrado
    \item Orientador/Professor
\end{itemize}
Cada pesquisador possui uma linha de pesquisa principal,
contudo pode haver participação em outras linhas de pesquisa.\\
O trabalho desses pesquisadores na linha de pesquisa resulta
em "memórias", que podem ser dividídas em vídeos, fotos e artigos.\\
Além das memórias, todos os pesquisadores participam de vários tipos
de reuniões agendadas:
\begin{itemize}
    \item Reunião exclusiva para líderes das linahs de pesquisa
    \item Reuniões específicas de cada Núcleo
    \item Reuniões gerais com todos os Núcleos
    \item Reuniões excepcionais
\end{itemize}
Outrossim os pesquisadores precisam entregar relatórios mensais,
trimestrais e anuais contemplando as soluções desenvolvidas até tal ponto.

\subsection{Consultas}
O sistema deve possuir esses diferenciais:
\begin{enumerate}
    \item Tornar as "memórias" públicas
    \item Expor dados informativos sobre os Núcleos, linhas de pesquisa e pesquisadores
    \item Restringir a criação de relatórios e "memórias" aos pesquisadores
    \item Restringir o agendamento de reuniões aos administradores e pesquisadores
    \item Permitir o cadastro de pesquisadores e montagem do relatórios apenas aos administradores
\end{enumerate}

\subsubsection{Média dos artigos publicados de cada pesquisador}

Nessa seção deve ser apresentada a lista de pessoas que participaram do levantamento de requisitos.


\subsubsection{“quais são os funcionários que trabalham em cada departamento?”}

Nessa seção deve ser apresentada a lista de pessoas que participaram do levantamento de requisitos.


\subsubsection{“listar todos os funcionários que trabalham no departamento de TI”}


\subsubsection{“quais são os funcionários que trabalham em cada departamento?”}


\subsubsection{“quais são os funcionários que trabalham em cada departamento?”}


\subsubsection{“quais são os funcionários que trabalham em cada departamento?”}


\subsubsection{“quais são os funcionários que trabalham em cada departamento?”}


\subsubsection{“quais são os funcionários que trabalham em cada departamento?”}


\subsection{Regras de negócio}
\label{sec:regrasdenegocio}

\begin{enumerate}
    \item \textbf{Número máximo de matrículas por semestre letivo}
          \begin{itemize}
              \item Em um semestre letivo um aluno não poderá se matricular em um número de disciplinas cujo o soma de créditos ultrapasse 18.
          \end{itemize}
    \item \textbf{Número máximo de alunos por turma}
          \begin{itemize}
              \item Uma oferta de disciplina não pode ter mais de 40 vagas para matrícula.
          \end{itemize}
    \item \textbf{Validação de créditos}
          \begin{itemize}
              \item Um aluno só poderá solicitar validação de uma disciplina caso nunca tenha se matriculado na mesma.
          \end{itemize}
\end{enumerate}


\section{Projeto Conceitual}

\begin{tabular}{|c|l|rc|}
    \hline
    jan & fev & mar & abr \\ \hline
    mai & jun & jul & ago \\ \cline{1-1} \cline{3-4}
    set & out & nov & dez \\ \hline \hline
\end{tabular}


\begin{tabular}{|l|ll|} \hline
    Pesquisador & \multicolumn{2}{|c|}{terça}      \\ \hline
    Data de nascimento                             \\
                & 10                          & 25 \\ \hline
\end{tabular}

% ----------------------------------------------------------
% Referências bibliográficas
% ----------------------------------------------------------
\bibliography{referencias}


\end{document}

