\documentclass[11pt]{../../classes/ifscarticle}
\usepackage{../../classes/ifscutils}
\usepackage[alf]{abntex2cite} % Citações padrão ABNT
\usepackage[nottoc]{tocbibind}
\renewcommand*{\theusecase}{UC-\thesection.\arabic{usecase}}
\usepackage{lipsum}

\AtBeginDocument{\thispagestyle{empty}}
\begin{document}

% ------------------------------------------------------------%
% Capa
% -------1-----------------------------------------------------%
\begin{center}

    {\large Universidade Estadual Norte Fluminense}\\[0.2cm] %0,2cm é a distância entre o texto dessa linha e o texto da próxima
    {\large Banco de Dados - Prof. Dra. Sahudy Montenegro González    }\\[0.2cm] % o comando \\ "manda" o texto ir para próxima linha
    {\large Ciência da Computação}\\[5.2cm]


    % Título
    {\Huge \bfseries Grupo 18 - PEA Pescarte}

    \vspace{.5cm}

    % Subtítulo
    {\LARGE \bfseries Fase Intermediária}

    \vfill
\end{center}
\begin{tabbing}

\end{tabbing}

{\noindent \large \bfseries
Javier Ernesto Lopez Del Real
\\[.5em] Matheus de Souza Pessanha
}


\begin{flushright}
    Data de entrega: 29 de mar\c{c}o de 2021
\end{flushright}

\clearpage
\pagestyle{firstpage}


% ------------------------------------------------------------%
% Adicionando sumário
% ------------------------------------------------------------%
\tableofcontents
\clearpage

% ------------------------------------------------------------%
% Início do documento
% ------------------------------------------------------------%

\section{Introdução}
\label{sec:introducao}

O Projeto de Educação Ambiental(PEA) PESCARTE tem como sua principal finalidade a criação de uma rede social regional integrada por pescadores artesanais e por seus familiares, buscando, por meio de processos educativos, promover, fortalecer e aperfeiçoar a sua organização comunitária e a sua qualificação profissional, bem como o seu envolvimento na construção participativa e na implementação de projetos de geração de trabalho e renda.
\begin{figure}[ht]
    \centering
    \includegraphics[width=.5\linewidth]{figuras/logoPescarte}
    \caption{Logo do Pescarte}
    \label{fig:logolatex}
\end{figure}

\section{Descrição do problema}

O PEA PESCARTE é divido por quatro núcleos:
\begin{itemize}
    \item Núcleo A
    \item Núcleo B
    \item Núcleo C
    \item Núcleo D
\end{itemize}
Esses Núcleos são compostos por 21(Vinte e uma) linhas
de pesquisa que por sua vez são formadas por
diversos pesquisadores. Um pesquisador possui três tipos:
\begin{itemize}
    \item Pesquisador de Iniciação Científica
    \item Pesquisador de mestrado
    \item Coordenador
\end{itemize}
Cada pesquisador possui uma linha de pesquisa principal,
contudo pode haver participação em outras linhas de pesquisa.
O trabalho desses pesquisadores na linha de pesquisa resulta
em "memórias", que podem ser divididas em vídeos, fotos e artigos.\\
Além das memórias, todos os pesquisadores participam de vários tipos
de reuniões agendadas:
\begin{itemize}
    \item Reunião exclusiva para líderes das linhas de pesquisa
    \item Reuniões específicas de cada Núcleo
    \item Reuniões gerais com todos os Núcleos
    \item Reuniões excepcionais
\end{itemize}
Outrossim, os pesquisadores precisam entregar relatórios mensais,
trimestrais e anuais contemplando as soluções desenvolvidas até tal ponto.\\
\clearpage
O sistema deve possuir esses diferenciais:

\begin{enumerate}
    \item Tornar as "memórias"\ públicas
    \item Expor dados informativos sobre os Núcleos, linhas de pesquisa e pesquisadores
    \item Consulta confirmação de email de usuário
    \item Verificar o papel do usuário
    \item Permitir o cadastro de pesquisadores e montagem do relatórios apenas aos administradores
\end{enumerate}

\subsection{Consultas}


O sistema deve realizar esses consultas:
\begin{enumerate}

    \item Média das memórias artigos publicados de cada pesquisador
    \item Total de linhas de pesquisa por pesquisador
    \item Número total de Pesquisadores por núcleo
    \item Verificar os papéis de um usuário
    \item Verificar os participantes de uma reunião
    \item Quantos usuários não confirmaram o email
\end{enumerate}


\clearpage
\section{Projeto Conceitual}
\subsection{Modelo Entidade-Relacionamento}
\begin{figure}[h]
    \centering
     \includegraphics[width=19cm]{figuras/Diagrama2.png}
    \caption{Diagrama}
    \label{fig:logolatex}
\end{figure}
\clearpage
\subsection{Tabela de metadados}
Tabelas de metadados com a descrição do tipo
de atributos por tipo-entidade e suas restrições.

\begin{table}[h]
    \centering
    \vspace{0.5cm}
    \begin{tabular}{ |p{2,5cm}|p{3cm}|p{4cm}|p{5,3cm}|  }
        Tipo-Entidade & Atributo        & Tipo          & Restrições                                       \\ % Note a separação de col. e a quebra de linhas

        \hline
        Linhas \linebreak de Pesquisa
                      &                 &               &                                                  \\
                      & Descrição curta & Monovalorado  & Obrigatório                        \linebreak    \\
                      & Número          & Identificador & Obrigatório                         \linebreak   \\
                      & Descrição longa & Monovalorado  & Opcional, <= 280 caracteres           \linebreak \\
        \hline
    \end{tabular}
    \caption{Tabela de Tipo-Entidade das Linhas de pesquisa.}
\end{table}


\begin{table}[h]
    \centering
    \vspace{0.5cm}
    \begin{tabular}{ |p{2,5cm}|p{3cm}|p{4cm}|p{5,3cm}|  }
        Tipo-Entidade & Atributo         & Tipo          & Restrições                                \\ % Note a separação de col. e a quebra de linhas

        \hline
        Núcleo        &                  &               &                                           \\
                      & Nome             & Identificador & Obrigatório        \linebreak             \\
                      & Descrição Núcleo & Monovalorado  & Obrigatório, <= 400 caracteres \linebreak \\

        \hline
    \end{tabular}
    \caption{Tabela de Tipo-Entidade do Núcleo.}
\end{table}


\begin{table}[h]
    \centering
    \vspace{0.5cm}
    \begin{tabular}{ |p{2,5cm}|p{3cm}|p{4cm}|p{5,3cm}|  }
        Tipo-Entidade & Atributo & Tipo          & Restrições                                \\ % Note a separação de col. e a quebra de linhas

        \hline
        Memórias
                      &          &               &                                           \\
                      & Link     & Chave parcial & Obrigatório    \linebreak                 \\
                      & Tipos    & Monovalorado  & Video \linebreak Imagem \linebreak Artigo \\

        \hline
    \end{tabular}
    \caption{Tipos de atributos por tipo-entidade da Memória.}
\end{table}


\begin{table}[h]
    \centering
    \vspace{0.5cm}
    \begin{tabular}{ p{2,5cm}|p{3cm}|p{4cm}|p{5,3cm}|  }


        Tipo-Entidade & Atributo      & Tipo            & Restrições                                             \\ % Note a separação de col. e a quebra de linhas

        \hline
        % para uma linha horizontal
        Usuario
                      &               &                 &                                                        \\
                      & Cpf           & Identificador   & Obrigatório            \linebreak                      \\
                      & E-mail        & Chave candidata & Obrigatório        \linebreak                          \\
                      & Nome completo & Chave candidata & Obrigatório                        \linebreak          \\
                      & Data nasc     & Monovalorado    & Obrigatório                        \linebreak          \\
                      & Hash Senha    & Monovalorado    & Obrigatório                        \linebreak          \\
                      & Tipo          & Monovalorado    & Administrador\linebreak Pesquisador\linebreak Pescador \\

        \hline
    \end{tabular}
    \caption{Tabela de Tipo-Entidade do Usuario.}
\end{table}

\begin{table}[h]
    \centering
    \vspace{0.5cm}
    \begin{tabular}{ p{2,5cm}|p{3cm}|p{4cm}|p{5,3cm}|  }

        Tipo-Entidade & Atributo      & Tipo            & Restrições                                                     \\ % Note a separação de col. e a quebra de linhas
        \hline
        Pesquisador
                      &               &                 &                                                                \\
                      & Link Lattes   & Chave candidata & Obrigatório                                        \linebreak  \\
                      & Tipo de bolsa & Monovalorado    & Obrigatório                                        \linebreak  \\
                      & Minibiografia & Monovalorado    & Obrigatório, <= 280 caracteres                      \linebreak \\
        \hline
    \end{tabular}
    \caption{Tabela de Tipo-Entidade do Pesquisador.}
\end{table}


\begin{table}[h]
    \centering
    \vspace{0.5cm}
    \begin{tabular}{ p{2,5cm}|p{3cm}|p{4cm}|p{5,3cm}|  }


        Tipo-Entidade & Atributo & Tipo          & Restrições                                                      \\ % Note a separação de col. e a quebra de linhas
        \hline
        Relatório
                      &          &               &                                                                 \\
                      & Link     & Monovalorado  & Obrigatório                                          \linebreak \\
                      & Data     & Chave parcial & Obrigatório                                          \linebreak \\
                      & Tipo     & Monovalorado  & Mensal\linebreak Trimestral\linebreak Anual                     \\
        \hline
    \end{tabular}
    \caption{Tabela de Tipo-Entidade do Relatório}
\end{table}


\clearpage

% ----------------------------------------------------------
% Referências bibliográficas
% ----------------------------------------------------------
\bibliography{referencias}        %use a bibtex bibliography file refs.bib


\begin{itemize}
    \item \href{https://github.com/cciuenf/pea_pescarte/blob/main/doc_projeto/documentos/fluxograma_modulos.pdf}{Fluxograma Módulos PEA Pescarte }

\end{itemize}

\end{document}

